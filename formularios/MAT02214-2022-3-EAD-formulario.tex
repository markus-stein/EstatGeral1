%%%%%%%%%%%%%%%%%%%%%%%%%%%%%%%%%%%%%%%%%%%%%%%%%%%%%%%%%%%%%%%%%%%%%%%%%%%%%%%%%%%%%%%%%%%
%
% FORMULÁRIO ÁREA 3
%
%%%%%%%%%%%%%%%%%%%%%%%%%%%%%%%%%%%%%%%%%%%%%%%%%%%%%%%%%%%%%%%%%%%%%%%%%%%%%%%%%%%%%%%%%%%

\documentclass[10pt]{article}%
%\usepackage[latin1]{inputenc}%
\usepackage[utf8x]{inputenc}
\usepackage[brazil]{babel}%
\usepackage{epsfig}%
\usepackage{amsmath}%
\usepackage{amsfonts}%
\usepackage{mathrsfs}%
\usepackage{amssymb}%
\usepackage{graphicx}%
\usepackage{longtable}%
\usepackage{geometry, calc, color, setspace}%
\usepackage{indentfirst}%
\usepackage{wrapfig}%
%\usepackage{cite}
\usepackage[round,authoryear]{natbib}
%\usepackage[colorlinks=true]{hyperref}%
\usepackage{lscape}
\usepackage{booktabs}
%\usepackage{enumitem}
\usepackage{fourier} 
\usepackage{array}
\usepackage{makecell}

\def\us{\char`\_}

%\pagestyle{empty}

\geometry{a4paper, headsep=1.0cm, footskip=1cm, lmargin=2.5cm, rmargin=2.5cm,
         tmargin=2.5cm, bmargin=2.5cm, headheight=2.5cm}

\renewcommand{\baselinestretch}{1.2}  % Espa\c{c}o entre linhas
\renewcommand{\ra}[1]{\renewcommand{\arraystretch}{#1}}

\renewcommand\theadalign{bc}
\renewcommand\theadfont{\bfseries}
\renewcommand\theadgape{\Gape[4pt]}
\renewcommand\cellgape{\Gape[4pt]}

\def\Var{{\rm Var}\,}
\def\E{{\rm E}\,}
%%%%%%%%%%%%%%%%%%%%%%%%%%%%%%%%%%%%%%%%%%%%%%%%%%%%%%%%%%%%
%
%             In\'{\i}cio do documento
%
%%%%%%%%%%%%%%%%%%%%%%%%%%%%%%%%%%%%%%%%%%%%%%%%%%%%%%%%%%%%
\begin{document}


%%%%%%%%%%%%%%%%%%%%%%%%%%%%%%%%%%%%%%%%%%%%%%%%%%%%%%%%%%%%
\begin{center}
{\bf 
UNIVERSIDADE FEDERAL DO RIO GRANDE DO SUL\\
INSTITUTO DE MATEMÁTICA E ESTATÍSTICA\\
DEPARTAMENTO DE ESTATÍSTICA\\
MAT02219 - PROBABILIDADE E ESTATÍSTICA}\\
\vspace*{0.1cm}{\bf ÁREA 3}
\end{center}
\pagenumbering{arabic}
%%%%%%%%%%%%%%%%%%%%%%%%%%%%%%%%%%%%%%%%%%%%%%%%%%%%%%%%%%%%
%\vspace*{0.1cm}
%\begin{center}
%{\bf INSTRUÇÕES}
%\end{center}
%
%\begin{itemize}%\setlength{\itemsep}{+1mm}
%\item A prova é individual.
%\item A prova é sem consulta, \textbf{com exceção} do formulário que se encontra a seguir.
%\item Você pode utilizar uma calculadora para realizar as questões.
%\begin{itemize}
%\item No caso de uma divisão, você pode deixar o resultado na forma de frações.
%\end{itemize}
%\item Leia com atenção todas as questões antes de resolvê-las.
%\item As questões podem ser realizadas na ordem de sua preferência.
%\item O desenvolvimento das questões será considerado. Portanto, escreva de maneira legível o seu raciocínio, esclarecendo suas suposições para a solução das questões. Não esqueça de indicar claramente o número da questão que você está respondendo.
%\item Coloque o seu nome e número de cartão nas folhas de resolução.
%\item A prova começará as 16 hs 30 min e tem duração de uma hora e quarenta minutos.
%\end{itemize}
%%%%%%%%%%%%%%%%%%%%%%%%%%%%%%%%%%%%%%%%%%%%%%%%%%%%%%%%%%%%
%\newpage
%%%%%%%%%%%%%%%%%%%%%%%%%%%%%%%%%%%%%%%%%%%%%%%%%%%%%%%%%%%%%
%\begin{center}
%{\bf 
%UNIVERSIDADE FEDERAL DO RIO GRANDE DO SUL\\
%INSTITUTO DE MATEMÁTICA E ESTATÍSTICA\\
%DEPARTAMENTO DE ESTATÍSTICA\\
%MAT02219 - PROBABILIDADE E ESTATÍSTICA}\\
%\vspace*{0.5cm}
%{\bf PROVA 2 - 2017/2}
%\end{center}
%%%%%%%%%%%%%%%%%%%%%%%%%%%%%%%%%%%%%%%%%%%%%%%%%%%%%%%%%%%%
%\vspace*{0.5cm}
\begin{center}
{\bf FORMULÁRIO}
\end{center}

\textbf{Variáveis aleatórias}

% \begin{itemize}%\setlength{\itemsep}{+1mm}
% \item Seja $X$ uma {\bf variável aleatória (v.a.) discreta} ($X$ assume valores em $\{x_1,x_2,\ldots\}$), então o {\bf valor esperado} (média) e a {\bf variância} de $X$, são, respectivamente:
% $$
% \mu = \E[X] = \sum_{j=1}^{\infty}{x_j p(x_j)}\quad\mbox{e}\quad \sigma^2 = \Var[X] = \E[(X - \mu)^2] = \sum_{j=1}^{\infty}{(x_j - \mu)^2 p(x_j)} = \E[X^2] - \mu^2,
% $$
% em que $p(x_j) = \Pr(X = x_j)$ é a {\bf função de probabilidade} de $X$ e $\E[X^2] = \sum_{j=1}^{\infty}{x_j^2 p(x_j)}$.
% \begin{itemize}
% \item[$\bigstar$] Se $X$ {\bf assumir apenas um número finito de valores}, as expressões acima tornam-se $\E[X] = \sum_{j=1}^{n}{x_j p(x_j)}$, $\Var[X] = \sum_{j=1}^{n}{(x_j - \mu)^2 p(x_j)}$ e $\E[X^2] = \sum_{j=1}^{n}{x_j^2 p(x_j)}$.
% \end{itemize}
\item Se $X$ é {\bf v.a. contínua}, então o valor esperado (média) e a variância de $X$, são, respectivamente:
$$
\mu = \E[X] = \int_{-\infty}^{\infty}{xf(x)dx}\quad\mbox{e}\quad \sigma^2 = \Var[X] = \E[(X - \mu)^2] = \int_{-\infty}^{\infty}{(x - \mu)^2 f(x)dx} = \E[X^2] - \mu^2,
$$
em que $f(x)$ é a {\bf função densidade de probabilidade} (fdp) de $X$ e $\E[X^2] = \int_{-\infty}^{\infty}{x^2f(x)dx}$.
\begin{itemize}
\item[$\bigstar$] Se $X$ {\bf assumir apenas valores em um intervalo} $S_X \subset \mathbb{R}$, as expressões acima tornam-se $\E[X] = \int_{S_X}{xf(x)dx}$, $\Var[X] = \int_{S_X}{(x - \mu)^2f(x)dx}$ e $\E[X^2] = \int_{S_X}{x^2f(x)dx}$.
\end{itemize}
\item Seja $X$ uma variável aleatória, então $F(x) = \Pr(X\leq x)$ é a {\bf função de distribuição acumulada} (fda) de $X$.
\begin{itemize}
% \item[$\bigstar$] $\displaystyle{F(x) = \sum_{j:x_j\leq x}{p(x_j)}}$, se $X$ é v.a. discreta.
\item[$\bigstar$] $\displaystyle{F(x) = \int_{-\infty}^x{f(u)du}}$, se $X$ é v.a. contínua.
\end{itemize}
\end{itemize}


\textbf{Distribuições de probabilidade}

Na Tabela \ref{tab1} considere:
\begin{itemize}
% \item ${\rm C}_n^x = {n \choose x} = \frac{n!}{x! (n - x)!}$ é o número de subconjuntos de $x$ elementos diferentes de um conjunto de $n$ elementos diferentes.
% \item Para a distribuição Bernoulli e Binomial, $\pi$ representa a probabilidade de sucesso ($\{X = 1\}$).
\item Para as distribuições Normal e $t$-Student, $\pi$ representa o valor $3,14\ldots$.
\item Para as distribuições Qui-quadrado e $t$-Student, $\Gamma(u) = \int_0^{\infty}{x^{u-1}e^{-x}dx}$.
\end{itemize}
\begin{table*}[ht]\centering
% \ra{1.3}
\caption{Distribuição de probabilidade, média e variância.}
\label{tab1}
\begin{tabular}{@{}llcc@{}}
\hline
% \multicolumn{4}{c}{\bf Variáveis aleatórias discretas}\\
% Distribuição de $X$ & $\Pr(X = x)$ & $\E[X]$ & $\Var[X]$\\
% \hline
% ${\rm Uniforme}(1,k)$ & $\displaystyle{\frac{1}{k}, j = 1, \ldots, k.}$ & $\displaystyle{\frac{1+k}{2}}$ & $\displaystyle{\frac{k^2 - 1}{12}}$\\
% ${\rm Bernoulli}(\pi)$ & $\displaystyle{\pi^x (1 - \pi)^{1 - x}, x = 0,1.}$ & $\pi$ & $\pi(1 - \pi)$\\
% ${\rm Binomial}(n, \pi)$ & $\displaystyle{{n \choose x}\pi^x (1 - \pi)^{n - x}, x = 0,\ldots,n.}$ & $n\pi$ & $n\pi(1 - \pi)$\\
% %\midrule
% %$\mbox{Geométrica}(p)$ &  $\displaystyle{(1-p)^{j-1}p, j = 1,2,\ldots.}$ & $\displaystyle{\frac{1}{p}}$ & $\displaystyle{\frac{1-p}{p^2}}$\\
% $\mbox{Hipergeométrica}(n,N,N_1)$ & $\displaystyle{\frac{{\rm C}_{N_1}^x {\rm C}_{N_2}^{n-x}}{{\rm C}_N^n}, x = 0, 1, \ldots, n.}$& $\displaystyle{n\frac{N_1}{N}}$ & $\displaystyle{n\frac{N_1}{N}\frac{N_2}{N}}\left(\frac{N-n}{N-1}\right)$\\
% ${\rm Poisson}(\lambda)$ & $\displaystyle{\frac{e^{-\lambda}\lambda^x}{x!}, x = 0, 1, \ldots.}$& $\lambda$ & $\lambda$\\
% \hline
% \midrule
\multicolumn{4}{c}{\bf Variáveis aleatórias contínuas}\\
\midrule
${\rm Uniforme}(\alpha,\beta)$ & $\displaystyle{\frac{1}{\beta-\alpha}, \alpha\leq x\leq \beta.}$ & $\displaystyle{\frac{\alpha + \beta}{2}}$ & $\displaystyle{\frac{(\beta - \alpha)^2}{12}}$\\
\midrule
${\rm Exponencial}(\lambda)$ &  $\displaystyle{\lambda e^{-\lambda x}, x > 0, \lambda > 0.}$ & $\displaystyle{\frac{1}{\lambda}}$ & $\displaystyle{\frac{1}{\lambda^2}}$\\
\midrule
${\rm Normal}(\mu, \sigma^2)$ & $\displaystyle{\frac{1}{\sqrt{2\pi\sigma^2}}e^{-\frac{(x-\mu)^2}{2\sigma^2}}, -\infty < x < \infty, -\infty < \mu < \infty,  \sigma^2 > 0.}$& $\mu$ & $\sigma^2$\\
\midrule
$\mbox{Normal}(0, 1)$ & $\displaystyle{\frac{1}{\sqrt{2\pi}}e^{-\frac{x^2}{2}}, -\infty < x < \infty.}$& $0$ & $1$\\
\midrule
${\rm Qui-quadrado}(\nu)$ & $\displaystyle{\frac{1}{2^{\nu/2}\Gamma(\nu/2)}x^{\frac{\nu}{x} - 1}e^{\frac{x}{2}}, x \geq 0, \nu > 0.}$& $\nu$ & $2\nu$\\
\midrule
$t{\rm -Student}(\nu)$ & $\displaystyle{\frac{\Gamma((\nu+1)/2)}{\Gamma(\nu/2)\sqrt{\pi\nu}}\left(1+\frac{x^2}{\nu}\right)^{-(\nu+1)/2}, -\infty < x < \infty, \nu > 0.}$& $0$ ($\nu > 1$)& $\frac{\nu}{\nu - 2}$ ($\nu > 2$)\\
\end{tabular}
\end{table*}

\textbf{Correlação e regressão linear}

\begin{itemize}
\item O coeficiente de correlação linear amostral entre $X$ e $Y$ é dado por $\displaystyle{r = \frac{S_{XY}}{\sqrt{S_{XX}S_{YY}}}}$, em que 
$$
S_{XX} = \sum_i{x_i^2} - (\sum_i{x_i})^2/n, \qquad S_{YY} = \sum_i{y_i^2} - (\sum_i{y_i})^2/n \qquad \mbox{ e } \quad  S_{XY} = \sum_i{x_iy_i} - [(\sum_i{x_i})(\sum_i{y_i})]/n.
$$
\begin{itemize}
\item[$\bigstar$] Seja $\rho$ o coeficiente de correlação linear populacional, para testar $H_0: \rho = 0 \times H_A: \rho \neq 0$, é usada a estatística de teste $T = \frac{r\sqrt{n - 2}}{\sqrt{1 - r^2}}$, que tem distribuição $t_{n-2}$.
\end{itemize}
\item Considere o modelo de regressão linear $Y = \beta_0 + \beta_1 X + \varepsilon$. Os estimadores de $\beta_0$ e $\beta_1$, denotados por $b_0$ e $b_1$, são dados por
$$
b_1 = \frac{\sum_i{x_iy_i} - [(\sum_i{x_i})(\sum_i{y_i})]/n}{\sum_i{x_i^2} - (\sum_i{x_i})^2/n}\ \mbox{e}\ b_0 = \bar{Y} - b_1\bar{X}.
$$
\begin{itemize}
\item[$\bigstar$] Para testar $H_0: \beta_1 = 0 \times H_A: \beta_1 \neq 0$, é usada a estatística de teste $T = \frac{b_1}{S_{b1}}$, em que $S_{b1} = \sqrt{\frac{S^2}{S_{XX}}}$ e $S^2 = \frac{S_{YY} - b_1S_{XY}}{n - 2}$; $T$ tem distribuição $t_{n-2}$.
\end{itemize}
\end{itemize}
\end{document}
